\chapter*{Abstract}

Compressed Sensing, also known as Compressive Sampling, is a 
novel framework that exploits the sparsity of a signal in some
known basis. Recent theoretical proofs show that if a signal is 
sparse in a known basis then it is possible to reconstruct
it via a small number of measurements. Furthermore, the signal can  
be accurately reconstructed using efficient algorithms. In this project we apply
 this technique in the Image reconstruction problem of Radio astronomy which is 
hampered by poor sampling and noisy Fourier measurements on aperture plane. The 
theory of Compressed Sensing demonstrates that such measurements may actually 
suffice for accurate reconstruction. Also, Compressed Sensings algorithms offers 
significant improvement over mainstream deconvolution algorithms used in radio 
interferometry such as CLEAN and MEM.
