\chapter{Summary and Conclusion}
\label{c:conclusion}

We have evaluated a new frame work called Compressed Sensing for image reconstruction
problem in Radio Interferometry. We have explored the class of algorithms
called Proximal Gradient methods for efficient algorithms suitable
for Radio-interferometric data. Finally we concluded that ISTA and FISTA
algorithms along with their variants gives better performance and shows significant improvement 
in image visual quality over mainstream deconvolution algorithms like CLEAN. 
Further analysis, and work by other authors show that FISTA and its 
variants are capable of competing favourably with traditional image reconstruction methods 
in radio astronomy in normal scenarios, and will do much better in special cases (like 3D imaging,
multi-frequency synthesis etc). We have, though a toy model, investigated and identified the core
problems involved in such an exercise, and have gone on to implement a working code for applying compressed
sensing to radio interferometric data from the GMRT. Along with collaborators at NCRA, we are
currently involved in solving the remaining outstanding issues - namely, choosing an optimal penalty parameter,
fidelity of flux reconstruction, and computational efficiency for large data sets. The code, which is a result
of this project, will be further developed at NCRA into a full-fledged pipeline for the GMRT data, over the coming
months. Possibilities for future work include the following:
\begin{enumerate}

\item  To estimate optimal value of penalty parameter
\item  To solve full Van Cittert–Zernike equation.
\item  To create a library of basis matrix.
\item  Modify the algorithm for gridded Visibilities.
\item  Modify the C code for all the above cases.

\end{enumerate}
